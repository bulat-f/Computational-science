\section{Вычисление при помощи ряда Тейлора} Разложим функцию ошибок в ряд Тейлора в окрестности точки $x$.
\begin{equation}
	\label{taylor}
	erf(x) = \frac{2}{\sqrt{\pi}}\sum^{\infty}_{n = 0}(-1)^n \frac{x^{2n+1}}{n!(2n+1)}
\end{equation}
\begin{theorem}
	Ряд ~(\ref{taylor}) сходится на всей числовой прямой.
\end{theorem}
\begin{proof}
	Докажем, что вычисление функции ошибок действительно возможно с использованием ряда. Как известно из курса математического анализа, ряд Тейлора для функции $e^x$ имеет вид:
	$$
		e^x = \sum\limits_{n=0}^{\infty} \frac{x^n}{n!}
	$$
	
	Причем радиус сходимости этого ряда бесконечен. Подставляя в эту формулу $x = -t^2$
	$$
		e^{-t^2} = \sum\limits_{n=0}^{\infty} \frac{(-1)^nt^{2n}}{n!}
	$$
	и интегрируя по $t$ от $0$ до $x$,
	$$
		\int\limits_{0}^{x} e^{-t^2} dt= \sum\limits_{n=0}^{\infty} \frac{(-1)^n}{n!}\int\limits_{0}^{x} t^{2n} dt = \sum\limits_{n=0}^{\infty} \frac{(-1)^nx^{2n+1}}{n!(2n+1)}
	$$
	
	Найдем радиус сходимости $R$ по формуле
	$$
		\frac{1}{R} = \lim\limits_{i \to \infty} \left| \frac{a_{i+1}}{a_i} \right|
	$$
	
	Легко находится, что $R = +\infty$.
\end{proof}

Следует заметить, что вычисление членов ряда ведет к потере точности и замедляет вычисления. Поэтому гораздо практичнее вычислять $i+1$-ый член ряда с помощью $i$-того, пользуясь соотношением. Причем в данном примере будет
$$
	q_i = \frac{a_{i+1}}{a_i} = \frac{-x^2(2n+1)}{(n+1)(2n+3)}
$$

Учтем, что ряд является знакопеременным, а значит к нему нужно применить теорему  Лейбница, гласящую, что остаток ряда не превосходит его первого отброшенного члена по абсолютной величине:
$$
	\left| \sum\limits_{n=N}^{+\infty} (-1)^nc_n \right| < \left| c_N \right|, \textit{при } c_N > 0.
$$
Протабулируем $erf(x)$ на отрезке [0, 2] на 10 узлах с точностью $10^{-6}$, основываясь на ряде Тейлора:\\
\begin{tabular}{ccccccccccc}
\hline
$0.00$&$0.20$&$0.40$&$0.60$&$0.80$&$1.00$&$1.20$&$1.40$&$1.60$&$1.80$&$2.00$\\
$0.000000$&$0.222703$&$0.428392$&$0.603856$&$0.742101$&$0.842701$&$0.910314$&$0.952285$&$0.976348$&$0.989091$&$0.995322$\\
\end{tabular}