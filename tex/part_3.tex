\section{Производная интерполяционного полинома}
Сравним $L’_n$ с $erf’(x)$, выбирая узлы интерполяции равноотстоями и как корни полиномов Чебышева. В качестве истинного значения производной можно взять производную от разложения функции в ряд Тейлора. Но, т.\,к. функция определяется как интеграл по верхнему пределу можно
$$
	erf’(x) =  \frac{2}{\sqrt{\pi}}e^{-t^2}dt
$$

А производная от интерполяционного полинома Лагранжа вычисляется следующим образом
$$
	L’_n = \sum\limits_{k=0}^{n} f_k\cdot l_k(x),
$$
где
$$
	l_k(x) \sum\limits_{j=0 j\neq k}^{n}\frac{\prod\limits_{i=0}^{n}(x-x_i)}{(x-x_k)(x-x_j)\prod\limits_{i=0 i\neq k}^{n}(x_k-x_j)}
$$


\subsection{Погрешности в точках, при выборе разных узлов интерполяции}
\input{derivative_errors}

\subsection{Повидение максимальной погрешности, при изменении количества узлов интерполяции}
\input{derivative_of_lagrange_max_error}