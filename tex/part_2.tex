\section{Интерполяционный полином Лагранжа}

Построим интерполяционный полином Лагранжа, приближающий $erf(x)$
$$
	L_n(x)=\sum\limits_{i=0}^{n}f(x_i)\prod\limits_{i \neq j, j = 0}^{n} \frac{x - x_j}{x_i-x_j}
$$

И вычислим погрешность интерполяции
$$
	\varepsilon_n = \max\limits_{x \in (a, b)} \varepsilon(x),
$$
где
$$
	\varepsilon(x)	= \left| erf(x) - L_n(x)\right|.
$$

\subsection{Погрешности в точках, при выборе разных узлов интерполяции}
\input{lagrange_errors}

\subsection{Повидение максимальной погрешности, при изменении количества узлов интерполяции}
\input{lagrange_max_error}

\subsection{Вывод}

В случае с равномерно распределенными узлами, при повышении степени интерполяционного полинома, ошибка начинает быстро возрастать. Это говорит о том, что при большом количестве узлов интерполирования этот интерполяционный полином не применим. Как мы видим из экспериментальных данных, погрешность начинает возрастать при степени полинома равной 35. А при степени полинома 45 она стремится к тысячам. В случае с Чебышевскими узлами интерполирования, тоже происходит возрастание погрешности. Но в данном случае процесс более плавный. При возрастании степени полинома, рост погрешности незначителен. Как мы видим из экспериментальных данных, погрешность начинает возрастать лишь при степени полинома равной 666. Наиболее устойчивое решение лежит в интервале от 380 до 644 узлов интерполирования. Этот метод более устойчив при использовании узлов Чебышева. То же мы самое мы видим и для  полинома Лагранжа в форме Ньютона.